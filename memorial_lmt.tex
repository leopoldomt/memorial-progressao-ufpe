%% Garcia's latex template for Activities Drescriptive Memorial Report
%% Version 0.2
%% (c) 2015 Vinicius Cardoso Garcia (vcg@cin.ufpe.br)
%%
%% This document is based on latex template from Prof. Daniel Cunha.
%%
%% Reference commands. Use the following commands to make references in your
%% text:
%%          \figref  -- for Figure reference
%%          \tabref  -- for Table reference
%%          \eqnref  -- for equation reference
%%          \chapref -- for chapter reference
%%          \secref  -- for section reference
%%          \appref  -- for appendix reference
%%          \axiref  -- for axiom reference
%%          \conjref -- for conjecture reference
%%          \defref  -- for definition reference
%%          \lemref  -- for lemma reference
%%          \theoref -- for theorem reference
%%          \corref  -- for corollary reference
%%          \propref -- for proprosition reference
%%          \pgref   -- for page reference
%%
%%          Example: See \chapref{chap:introduction}
%%%%%%%%%%%%%%%%%%%%%%%%%%%%%%%%%%%%%%%%%%%%%%%%%%%%%%%%%%%%%%%%%%%%%%%%%%%%%%%

\documentclass[a4paper,oneside,10pt]{article}

\usepackage{graphicx}
\usepackage{amsmath,amstext,amssymb,amsfonts}
\usepackage[none]{hyphenat}
\usepackage{fancyhdr}
\usepackage{cite}
\usepackage{indentfirst}
%\usepackage{path}
\usepackage[usenames, dvipsnames]{color}

\usepackage{pdfsync}
\usepackage{url}
\usepackage{setspace}
\usepackage[latin1,utf8]{inputenc}
\usepackage[T1]{fontenc}
\usepackage{colortbl}
\newcommand{\SetRowColor}[1]{\noalign{\gdef\RowColorName{#1}}\rowcolor{\RowColorName}}

\definecolor{MyRed}{rgb}{1,0.2,0.1}
\definecolor{light-gray}{gray}{0.95}
\definecolor{gray}{gray}{0.6}

\usepackage{booktabs}
\usepackage{ctable}
\usepackage{setspace}
\usepackage[multidot]{grffile}
\usepackage[final]{pdfpages}

\usepackage{titlesec}

\usepackage[toc,page]{appendix}
\newcommand{\appref}[1]{\@appendixname~\ref{#1}\xspace}
%\usepackage{xifthen}

\usepackage[bookmarks,colorlinks,pdfpagelabels,
pdftitle={Memorial Descritivo de Atividades}, pdfauthor={Leopoldo Motta Teixeira},
pdfsubject={Solicita\c{c}\~{a}o de progresss\~{a}o funcional docente de Adjunto A N\'{\i}vel 1 para Adjunto 1 apresentada \`{a} Comiss\~{a}o de Avalia\c{c}\~{a}o de Progress\~{a}o Horizontal do Centro de Inform\'{a}tica da Universidade Federal de Pernambuco.},
pdfcreator={Leopoldo Motta Teixeira}, pdfkeywords={Progressão, Adjunto, UFPE, CIn, Docente, Horizontal}]{hyperref}

\newcommand{\otoprule}{\midrule[\heavyrulewidth]}

% Defini\c{c}\~{a}o de margens
\setlength{\textwidth}{16cm} %
\setlength{\textheight}{23cm} %
\setlength{\oddsidemargin}{0cm} %
\setlength{\evensidemargin}{0cm} %
\setlength{\topmargin}{0cm}

\renewcommand{\abstractname}{Resumo}
\renewcommand{\contentsname}{\'{I}ndice Anal\'{\i}tico}
\renewcommand{\refname}{Refer\^{e}ncias}
\renewcommand{\appendixname}{Ap\^{e}ndice}
\renewcommand{\tablename}{Tabela}

%% Formata\c{c}\~{a}o do cabe\c{c}alho e rodap\'{e}
\lhead{\footnotesize Memorial Descritivo de Atividades} %
\rhead{\footnotesize \emph{Leopoldo Motta Teixeira}} %
\chead{} %
\cfoot{} %
\lfoot{\footnotesize\nouppercase\leftmark} %
\rfoot{\bfseries\thepage}
\renewcommand{\footrulewidth}{0.1pt}

% Comando para inserir n\'{u}mero de documento
\newcounter{document}%[section]
\setcounter{document}{0}
\renewcommand\theenumi{\arabic{section}.\arabic{enumi}}
\newcommand\Doc{{\addtocounter{document}{1}\mbox{\sffamily\bfseries [Doc. \arabic{document}]}}}

% Comando para repetir um n\'{u}mero de documento j\'{a} citado
% \mbox{\sffamily{\bfseries{[Doc. XX]}}}

%% Alternativa na edi\c{c}\~{a}o dos comandos.
%% Comando para inserir n\'{u}mero de documento
% \newcommand\thedocument{%
%    \ifthenelse{\arabic{subsection}=0}
%      {\thesection.\arabic{document}}
%      {\thesubsection.\arabic{document}}}
% \newcounter{document}[section]
% \setcounter{document}{0}
% \renewcommand\theenumi{\arabic{section}.\arabic{enumi}}
% \ifthenelse{\arabic{subsection} = 0}{\newcommand\Doc{{\stepcounter{document}\bfseries [Doc. \arabic{section}.\arabic{document}]}}}{\newcommand\Doc{{\stepcounter{document}\bfseries [Doc. \arabic{section}.\arabic{subsection}.\arabic{document}]}}}
% %\newcommand\Doc{{\addtocounter{document}{1}\mbox{\bfseries [Doc. \arabic{document}]}}}


% Ambiente para centralizar vertical
\newenvironment{vcenterpage}
     {\newpage\vspace*{\fill}}
     {\vspace*{\fill}\par\pagebreak}

\sloppy

\pagestyle{fancy}

\setcounter{secnumdepth}{4}

\begin{document}

\begin{titlepage}

\vspace{-5.0cm}

%\begin{figure}[!htb]
% \centering{\includegraphics[width=0.5\textwidth]{cin.pdf}}
% \label{fig:UFPE_logo}
%\end{figure}

\begin{center}
\vspace{1cm}
%{\huge \textsf{Solicita\c{c}\~{a}o de Progress\~{a}o Funcional Docente}} \\[1cm]
\rule{1.0\textwidth}{1pt} \\ [0.5cm]
{\Huge \textbf{\textsf{Memorial Descritivo de Atividades}}} \\
\rule{1.0\textwidth}{1pt} \\
\vspace{2cm}

\doublespacing
{\Large \textsf{Solicita\c{c}\~{a}o de progresss\~{a}o funcional docente de \textbf{Adjunto A N\'{\i}vel 1 para Adjunto 1} apresentada \`{a} Comiss\~{a}o de Avalia\c{c}\~{a}o de Progress\~{a}o Horizontal do Centro de Inform\'{a}tica da Universidade Federal de Pernambuco}}\\
\vspace{1.5cm}
{\LARGE \textsf{Solicitante: \textbf{Leopoldo Motta Teixeira}}}\\
\vspace{0.5cm}
{\Large \textsf{SIAPE: \textbf{1114746}}} \\
\vspace{0.5cm}
{\Large \textsf{Per\'{\i}odo: \textbf{25/08/2014 - 25/08/2017}}} \\

\vspace{2.0cm}

\normalsize \textsf{Novembro de 2019}

\end{center}
\thispagestyle{empty}
\end{titlepage}


\tableofcontents
%\include{Lista_Anexos} \cleartooddpage[\thispagestyle{empty}]

%%%%%%%%%%%%%%%%%%%%%%%%%%%%%%%%%%%%%%%%%%%%%%%%%%%%%%%%%%%%%%%%%%%%%%%%%%%%%%%
% APRESENTA\c{C}\~{A}O
%%%%%%%%%%%%%%%%%%%%%%%%%%%%%%%%%%%%%%%%%%%%%%%%%%%%%%%%%%%%%%%%%%%%%%%%%%%%%%%

\newpage
\section*{Apresenta\c{c}\~{a}o}
\vspace{0.3cm}

\begin{onehalfspace}

Memorial apresentado por \textbf{Leopoldo Motta Teixeira}, Professor Adjunto A Nível 1, lotado no Departamento de Sistemas de Computação do Centro de Inform\'{a}tica (CIn) da Universidade Federal de Pernambuco (UFPE), para avalia\c{c}\~{a}o de desempenho acad\^{e}mico, para fins de acesso por Progress\~{a}o Acelerada \`{a} classe de Professor Adjunto 1.

O presente memorial relata as atividades desempenhadas no per\'{\i}odo de \textbf{25 de Agosto de 2014 a 25 de Agosto de 2017} e foi elaborado com base nas diretrizes estabelecidas nas Resolu\c{c}\~{o}es 04/2008 e 02/2011 do Conselho Universit\'{a}rio (CONSUN) da UFPE e na Portaria 554, de 20/06/2013, do Minist\'{e}rio da Educa\c{c}\~{a}o (MEC). Os documentos comprobat\'{o}rios referenciados neste memorial est\~{a}o organizados em volumes anexos devidamente numerados. Por fim, os Anexos que se referem a artigos publicados em confer\^{e}ncias e peri\'{o}dicos cont\'{e}m apenas a primeira p\'{a}gina do trabalho e o email informando a aceita\c{c}\~{a}o do trabalhao para publica\c{c}\~{a}o (quando pertinente).

\end{onehalfspace}

%\subsection*{Esclarecimentos}
%
%Aqui nesta seção especial podem ser listadas algumas observa\c{c}\~{o}es acerca da documenta\c{c}\~{a}o comprobat\'{o}ria para dirimir poss\'{\i}veis d\'{u}vidas da Comiss\~{a}o Avaliadora. A numera\c{c}\~{a}o indicada corresponde \`{a} numera\c{c}\~{a}o do Anexo.
%
%\begin{itemize}
%
%\item [18.] Artigo dispon\'{\i}vel em \verb"dx.doi.org/10.14209/sbrt.2013.182".
%\item [19.] Lista de artigos aceitos dispon\'{\i}vel em \verb"http://www.sbrt.org.br/sbrt2012/artigos.html". O artigo em quest\~{a}o possui identifica\c{c}\~{a}o 98879.
%\item E assim por diante.
%
%\end{itemize}


%%%%%%%%%%%%%%%%%%%%%%%%%%%%%%%%%%%%%%%%%%%%%%%%%%%%%%%%%%%%%%%%%%%%%%%%%%%%%%%
% Grupo 1 - Atividades de Ensino
%%%%%%%%%%%%%%%%%%%%%%%%%%%%%%%%%%%%%%%%%%%%%%%%%%%%%%%%%%%%%%%%%%%%%%%%%%%%%%%
\newpage
\section{Atividades de Ensino}

%A seguir, listo as atividades de ensino que realizei no per\'{\i}odo, separadas por subgrupo, conforme rege o documento tal.

%%%%%%%%%%%%%%%%%%%%%%%%%%%%%%%%%%%%%%%%%%%%%%%%%%%%%%%%%%%%%%%%%%%%%%%%%%%%%%%
% Subgrupo 1.1 - Orienta\c{c}\~{o}es e Co-Orienta\c{c}\~{o}es
%%%%%%%%%%%%%%%%%%%%%%%%%%%%%%%%%%%%%%%%%%%%%%%%%%%%%%%%%%%%%%%%%%%%%%%%%%%%%%%
\subsection{Orienta\c{c}\~{o}es e Co-Orienta\c{c}\~{o}es}
\vspace{0.3cm}

%\subsubsection{Orienta\c{c}\~{a}o de Teses de Doutorado Conclu\'{i}das}
%\vspace{0.3cm}
%
%\begin{enumerate}
%\renewcommand{\labelenumi}{{\large\bfseries\arabic{enumi}.}}
%
%\item       \textbf{Aluno:} Jack Allan Bauer \mbox{\sffamily{\bfseries{[Doc. \ref{advisor:phd-concluidas}]}}} \\
%            \textbf{T\'{\i}tulo da Tese:} Salvando o mundo em 24 horas.\\
%            \textbf{Data da Defesa:} 6 de Novembro de 2001 \\
%            \textbf{Institui\c{c}\~{a}o:} Universidade Federal de Pernambuco.
%
%\end{enumerate}

%------------------------------------------------------------------------------

%\subsubsection{Co-Orienta\c{c}\~{a}o de Teses de Doutorado Conclu\'{i}das}
%\vspace{0.3cm}
%
%\begin{enumerate}
%\renewcommand{\labelenumi}{{\large\bfseries\arabic{enumi}.}}
%
%\item       \textbf{Aluno:} Thomas A. Anderson \mbox{\sffamily{\bfseries{[Doc. \ref{co-advisor:phd-concluidas}]}}}\\
%            \textbf{T\'{\i}tulo da Tese:} The Matrix \\
%            \textbf{Data da Defesa:} 31 de Março de 1999 \\
%            \textbf{Institui\c{c}\~{a}o:} Universidade Federal de Pernambuco.
%
%\end{enumerate}

%------------------------------------------------------------------------------

\subsubsection{Orienta\c{c}\~{a}o de Teses de Doutorado em Andamento}
\vspace{0.3cm}

Nada a declarar neste subgrupo.
%\begin{enumerate}
%\renewcommand{\labelenumi}{{\large\bfseries\arabic{enumi}.}}
%
%\item       \textbf{Aluno:} Morpheus Laurence Fishburne) \mbox{\sffamily{\bfseries{[Doc. \ref{advisor:phd-andamento}]}}} \\
%            \textbf{T\'{\i}tulo da Tese:} Nabucodonozor in The Matrix \\
%            \textbf{Data de In\'{\i}cio:} Mar\c{c}o de 2001 \\
%            \textbf{Institui\c{c}\~{a}o:} CIn/UFPE
%
%\end{enumerate}

%------------------------------------------------------------------------------

\subsubsection{Co-Orienta\c{c}\~{a}o de Teses de Doutorado em Andamento}
\vspace{0.3cm}

% Nada a declarar neste subgrupo.
\begin{enumerate}
\renewcommand{\labelenumi}{{\large\bfseries\arabic{enumi}.}}

\item       \textbf{Aluno:} Thiago Mael de Castro \mbox{\sffamily{\bfseries{[Doc. \ref{co-advisor:phd-andamento}]}}}\\
            \textbf{T\'{\i}tulo da Disserta\c{c}\~{a}o:} A Machine-Verified Theory of Commuting Strategies for Product-Line Reliability Analysis\\
            \textbf{Tipo:} Acadêmico \\%| Profissional\\
            \textbf{Data da In\'{\i}cio:} Março de 2017\\
            \textbf{Institui\c{c}\~{a}o:} Departamento de Ciência da Computação - Universidade de Brasília
\end{enumerate}

%------------------------------------------------------------------------------

\subsubsection{Orienta\c{c}\~{a}o de Disserta\c{c}\~{o}es de Mestrado Conclu\'{i}das}
\vspace{0.3cm}

Nada a declarar neste subgrupo.
%\begin{enumerate}
%\renewcommand{\labelenumi}{{\large\bfseries\arabic{enumi}.}}
%
%\item       \textbf{Aluno:}  Obi-Wan Kenobi \mbox{\sffamily{\bfseries{[Doc. \ref{advisor:msc-concluidas}]}}} \\
%            \textbf{T\'{\i}tulo da Disserta\c{c}\~{a}o:} A Vingança dos Sith\\
%            \textbf{Tipo:} Acadêmico \\%| Profissional\\
%            \textbf{Data da Defesa:} 19 de Maio de 2005\\
%            \textbf{Institui\c{c}\~{a}o:} CIn/UFPE.
%
%\item       \textbf{Aluno:} Darth Maul \mbox{\sffamily{\bfseries{[Doc. \ref{advisor:mprof-concluidas}]}}} \\
%            \textbf{T\'{\i}tulo da Disserta\c{c}\~{a}o:} Lord of the Sith\\
%            \textbf{Tipo:} Profissional\\
%            \textbf{Data da Defesa:} 19 de Maio de 2005\\
%            \textbf{Institui\c{c}\~{a}o:} CIn/UFPE.
%
%\end{enumerate}

%------------------------------------------------------------------------------

\subsubsection{Co-Orienta\c{c}\~{a}o de Disserta\c{c}\~{o}es de Mestrado Conclu\'{i}das}
\vspace{0.3cm}

% Nada a declarar neste subgrupo.

\begin{enumerate}
\renewcommand{\labelenumi}{{\large\bfseries\arabic{enumi}.}}

\item       \textbf{Aluno:} Thiago Mael de Castro \mbox{\sffamily{\bfseries{[Doc. \ref{co-advisor:msc-concluidas}]}}}\\
            \textbf{T\'{\i}tulo da Disserta\c{c}\~{a}o:} Estratégias Comutativas para Análise de Confiabilidade em Linhas de Produtos de Software\\
            \textbf{Tipo:} Acadêmico \\%| Profissional\\
            \textbf{Data da Defesa:} 18 de Novembro de 2016 \\
            \textbf{Institui\c{c}\~{a}o:} Departamento de Ciência da Computação - Universidade de Brasília

\item       \textbf{Aluno:} Fernando Chaves Benbassat \mbox{\sffamily{\bfseries{[Doc. \ref{co-advisor:msc-concluidas}]}}} \\
           \textbf{T\'{\i}tulo da Disserta\c{c}\~{a}o:} Evolu\c{c}\~{a}o Segura de Linhas de Produtos de Software: Cen\'{a}rios de Extra\c{c}\~{a}o de Features\\
           \textbf{Tipo:} Acadêmico \\%| Profissional\\
           \textbf{Data da Defesa:} 20 de Fevereiro de 2017 \\
           \textbf{Institui\c{c}\~{a}o:} Centro de Informática - Universidade Federal de Pernambuco
           
\item      \textbf{Aluno:} Gabriela Cunha Sampaio \mbox{\sffamily{\bfseries{[Doc. \ref{co-advisor:msc-concluidas}]}}} \\
           \textbf{T\'{\i}tulo da Disserta\c{c}\~{a}o:} Evolu\c{c}\~{a}o Parcialmente Segura de Linhas de Produtos de Software\\
           \textbf{Tipo:} Acadêmico \\%| Profissional\\
           \textbf{Data da Defesa:} 20 de Mar\c{c}o de 2017 \\
           \textbf{Institui\c{c}\~{a}o:} Centro de Informática - Universidade Federal de Pernambuco

\end{enumerate}

%------------------------------------------------------------------------------

\subsubsection{Orienta\c{c}\~{a}o de Disserta\c{c}\~{o}es de Mestrado em Andamento}
\vspace{0.3cm}

\begin{enumerate}
\renewcommand{\labelenumi}{{\large\bfseries\arabic{enumi}.}}

\item       \textbf{Aluno:} Alex Juvencio Costa \mbox{\sffamily{\bfseries{[Doc. \ref{advisor:msc-andamento}]}}}\\
            \textbf{T\'{\i}tulo da Disserta\c{c}\~{a}o:} Variabilidade em Aplicações de Cidades Inteligentes\\
            \textbf{Tipo:} Acadêmico \\%| Profissional\\
            \textbf{Data de In\'{\i}cio:} Agosto de 2016\\
            \textbf{Institui\c{c}\~{a}o:} Centro de Informática - Universidade Federal de Pernambuco

\item       \textbf{Aluno:} Gabriel Ibson de Souza \mbox{\sffamily{\bfseries{[Doc. \ref{advisor:msc-andamento}]}}}\\
            \textbf{T\'{\i}tulo da Disserta\c{c}\~{a}o:} Implementando refatorações para linhas de produtos de software\\
            \textbf{Tipo:} Acadêmico\\
            \textbf{Data de In\'{\i}cio:} Agosto de 2016\\
            \textbf{Institui\c{c}\~{a}o:} Centro de Informática - Universidade Federal de Pernambuco

\item       \textbf{Aluno:} Karine Galdino Maia Gomes \mbox{\sffamily{\bfseries{[Doc. \ref{advisor:msc-andamento}]}}}\\
            \textbf{T\'{\i}tulo da Disserta\c{c}\~{a}o:} Abordagens de verificação de refatorações em linhas de produtos de software\\
            \textbf{Tipo:} Acadêmico\\
            \textbf{Data de In\'{\i}cio:} Agosto de 2016\\
            \textbf{Institui\c{c}\~{a}o:} Centro de Informática - Universidade Federal de Pernambuco

\item       \textbf{Aluno:} Victor Laerte de Oliveira \mbox{\sffamily{\bfseries{[Doc. \ref{advisor:msc-andamento}]}}}\\
            \textbf{T\'{\i}tulo da Disserta\c{c}\~{a}o:} Estudo de caso de implementação e gerenciamento de variabilidade usando portlets\\
            \textbf{Tipo:} Acadêmico\\
            \textbf{Data de In\'{\i}cio:} Agosto de 2016\\
            \textbf{Institui\c{c}\~{a}o:} Centro de Informática - Universidade Federal de Pernambuco
            
\item       \textbf{Aluno:} Bruce Fabian Reis Albuquerque \mbox{\sffamily{\bfseries{[Doc. \ref{advisor:mprof-andamento}]}}}\\
            \textbf{T\'{\i}tulo da Disserta\c{c}\~{a}o:} Avaliação de conformidade de sites governamentais ao ePWG - Padrões Web em Governo Eletrônico\\
            \textbf{Tipo:} Profissional\\
            \textbf{Data de In\'{\i}cio:} Outubro de 2015\\
            \textbf{Institui\c{c}\~{a}o:} Centro de Informática - Universidade Federal de Pernambuco
            
\end{enumerate}

%------------------------------------------------------------------------------
\subsubsection{Co-Orienta\c{c}\~{a}o de Disserta\c{c}\~{o}es de Mestrado em Andamento}
\vspace{0.3cm}

Nada a declarar neste subgrupo.

% \begin{enumerate}
% \renewcommand{\labelenumi}{{\large\bfseries\arabic{enumi}.}}

% \item       \textbf{Aluno:} Fernando Chaves Benbassat \mbox{\sffamily{\bfseries{[Doc. \ref{co-advisor:msc-andamento}]}}} \\
%             \textbf{T\'{\i}tulo da Disserta\c{c}\~{a}o:} Tales of Suspense\\
%             \textbf{Tipo:} Acadêmico \\%| Profissional\\
%             \textbf{Data de Início:} Agosto de 2014 \\
%             \textbf{Institui\c{c}\~{a}o:} Centro de Informática - Universidade Federal de Pernambuco

% \item       \textbf{Aluno:} Gabriela Cunha Sampaio \mbox{\sffamily{\bfseries{[Doc. \ref{co-advisor:msc-andamento}]}}}\\
%             \textbf{T\'{\i}tulo da Disserta\c{c}\~{a}o:} Partially safe evolution of software product lines\\
%             \textbf{Tipo:} Acadêmico \\%| Profissional\\
%             \textbf{Data de In\'{\i}cio:} Março de 2015\\
%             \textbf{Institui\c{c}\~{a}o:} Centro de Informática - Universidade Federal de Pernambuco

% \item       \textbf{Aluno:} Thiago Mael de Castro \mbox{\sffamily{\bfseries{[Doc. \ref{co-advisor:msc-andamento}]}}}\\
%             \textbf{T\'{\i}tulo da Disserta\c{c}\~{a}o:} Estratégias Comutativas para Análise de Confiabilidade em Linhas de Produtos de Software\\
%             \textbf{Tipo:} Acadêmico \\%| Profissional\\
%             \textbf{Data de In\'{\i}cio:} Março de 2014\\
%             \textbf{Institui\c{c}\~{a}o:} Departamento de Ciência da Computação - Universidade de Brasília

% %\item       \textbf{Aluno:} Charles Francis Xavier \mbox{\sffamily{\bfseries{[Doc. \ref{co-advisor:mprof-andamento}]}}}\\
% %            \textbf{T\'{\i}tulo da Disserta\c{c}\~{a}o:} Magneto\\
% %            \textbf{Tipo:} Profissional\\
% %            \textbf{Data de In\'{\i}cio:} Setembro de 1963\\
% %            \textbf{Institui\c{c}\~{a}o:} Marvel Comics
% \end{enumerate}

%------------------------------------------------------------------------------

\subsubsection{Orienta\c{c}\~{a}o de Trabalhos de Conclus\~{a}o de Curso}
\vspace{0.3cm}

\begin{enumerate}
\renewcommand{\labelenumi}{{\large\bfseries\arabic{enumi}.}}

\item       \textbf{Aluno:} Antonio Alves Correia \mbox{\sffamily{\bfseries{[Doc. \ref{advisor:tcc}]}}}\\
            \textbf{Curso:} Ciência da Computação\\
            \textbf{T\'{\i}tulo da Monografia:} Uma ferramenta de refactoring para disciplinar anotações em linhas de produto de software\\
            \textbf{Data da Defesa:} 18 de Janeiro de 2016\\
            \textbf{Institui\c{c}\~{a}o:} Centro de Informática - Universidade Federal de Pernambuco

\item       \textbf{Aluno:} Vinícius Carneiro Pereira Souza \mbox{\sffamily{\bfseries{[Doc. \ref{advisor:tcc}]}}}\\
            \textbf{Curso:} Ciência da Computação\\
            \textbf{T\'{\i}tulo da Monografia:} Uma ferramenta leve de análise para descoberta estática de comunicações entre componentes de aplicações Android\\
            \textbf{Data da Defesa:} 21 de Janeiro de 2016\\
            \textbf{Institui\c{c}\~{a}o:} Centro de Informática - Universidade Federal de Pernambuco

\item       \textbf{Aluno:} João Paulo Tenório Trindade \mbox{\sffamily{\bfseries{[Doc. \ref{advisor:tcc}]}}}\\
            \textbf{Curso:} Ciência da Computação\\
            \textbf{T\'{\i}tulo da Monografia:} IncR: Ferramenta de Detecção Incremental de Comunicação entre Componentes Android\\
            \textbf{Data da Defesa:} 19 de Julho de 2016\\
            \textbf{Institui\c{c}\~{a}o:} Centro de Informática - Universidade Federal de Pernambuco

\item       \textbf{Aluno:} Luiz Antonio de Vasconcelos Filho \mbox{\sffamily{\bfseries{[Doc. \ref{advisor:tcc}]}}}\\
            \textbf{Curso:} Ciência da Computação\\
            \textbf{T\'{\i}tulo da Monografia:} Uma análise do uso de mapas open source em aplicações Android\\
            \textbf{Data da Defesa:} 13 de Fevereiro de 2017\\
            \textbf{Institui\c{c}\~{a}o:} Centro de Informática - Universidade Federal de Pernambuco

\end{enumerate}

%------------------------------------------------------------------------------

\subsubsection{Orienta\c{c}\~{a}o de Monitoria}
\vspace{0.3cm}

\begin{enumerate}
\renewcommand{\labelenumi}{{\large\bfseries\arabic{enumi}.}}

\item   \textbf{Aluno:} Júlio José de Oliveira Ribeiro Toscano de Brito \mbox{\sffamily{\bfseries{[Doc. \ref{advisor:monitoria}]}}} \\
        \textbf{Disciplina:}  Algoritmos e Estruturas de Dados (turma SI)\\
        \textbf{Curso:} Sistemas de Informação\\
        \textbf{Semestre:} 2014.2

\item   \textbf{Aluno:} Juliana Carvalho de Oliveira \mbox{\sffamily{\bfseries{[Doc. \ref{advisor:monitoria}]}}} \\
        \textbf{Disciplina:}  Algoritmos e Estruturas de Dados (turma SI)\\
        \textbf{Curso:} Sistemas de Informação\\
        \textbf{Semestre:} 2015.1

\item   \textbf{Aluno:} Pedro Vinícius Batista Clericuzi \mbox{\sffamily{\bfseries{[Doc. \ref{advisor:monitoria}]}}} \\
        \textbf{Disciplina:}  Algoritmos e Estruturas de Dados (turma SI)\\
        \textbf{Curso:} Sistemas de Informação\\
        \textbf{Semestre:} 2015.1

\item   \textbf{Aluno:} Valdi Ferreira do Nascimento Júnior \mbox{\sffamily{\bfseries{[Doc. \ref{advisor:monitoria}]}}} \\
        \textbf{Disciplina:}  Algoritmos e Estruturas de Dados (turma SI)\\
        \textbf{Curso:} Sistemas de Informação\\
        \textbf{Semestre:} 2015.1

\item   \textbf{Aluno:} Juliana Carvalho de Oliveira \mbox{\sffamily{\bfseries{[Doc. \ref{advisor:monitoria}]}}} \\
        \textbf{Disciplina:}  Algoritmos e Estruturas de Dados (turma SI)\\
        \textbf{Curso:} Sistemas de Informação\\
        \textbf{Semestre:} 2015.2

\item   \textbf{Aluno:} Pedro Vinícius Batista Clericuzi \mbox{\sffamily{\bfseries{[Doc. \ref{advisor:monitoria}]}}} \\
        \textbf{Disciplina:}  Algoritmos e Estruturas de Dados (turma SI)\\
        \textbf{Curso:} Sistemas de Informação\\
        \textbf{Semestre:} 2015.2

\item   \textbf{Aluno:} Valdi Ferreira do Nascimento Júnior \mbox{\sffamily{\bfseries{[Doc. \ref{advisor:monitoria}]}}} \\
        \textbf{Disciplina:}  Algoritmos e Estruturas de Dados (turma SI)\\
        \textbf{Curso:} Sistemas de Informação\\
        \textbf{Semestre:} 2015.2


\item   \textbf{Aluno:} José Augusto de Matos Almeida Júnior \mbox{\sffamily{\bfseries{[Doc. \ref{advisor:monitoria}]}}} \\
        \textbf{Disciplina:}  Algoritmos e Estruturas de Dados (turma SI)\\
        \textbf{Curso:} Sistemas de Informação\\
        \textbf{Semestre:} 2016.1

\item   \textbf{Aluno:} Lucas Serra da Cunha Assad \mbox{\sffamily{\bfseries{[Doc. \ref{advisor:monitoria}]}}} \\
        \textbf{Disciplina:}  Algoritmos e Estruturas de Dados (turma SI)\\
        \textbf{Curso:} Sistemas de Informação\\
        \textbf{Semestre:} 2016.1

\item   \textbf{Aluno:} Matheus Raz de Oliveira Leandro \mbox{\sffamily{\bfseries{[Doc. \ref{advisor:monitoria}]}}} \\
        \textbf{Disciplina:}  Algoritmos e Estruturas de Dados (turma SI)\\
        \textbf{Curso:} Sistemas de Informação\\
        \textbf{Semestre:} 2016.1

\item   \textbf{Aluno:} Gabriel Henrique Daniel da Silva \mbox{\sffamily{\bfseries{[Doc. \ref{advisor:monitoria}]}}} \\
        \textbf{Disciplina:}  Teoria e Implementação de Linguagens Computacionais (turma CC)\\
        \textbf{Curso:} Sistemas de Informação\\
        \textbf{Semestre:} 2016.2

\item   \textbf{Aluno:} Gabriel Henrique Daniel da Silva \mbox{\sffamily{\bfseries{[Doc. \ref{advisor:monitoria}]}}} \\
        \textbf{Disciplina:}  Teoria e Implementação de Linguagens Computacionais (turma CC)\\
        \textbf{Curso:} Sistemas de Informação\\
        \textbf{Semestre:} 2017.1

\item   \textbf{Aluno:} José Murilo Sodré da Mota Filho \mbox{\sffamily{\bfseries{[Doc. \ref{advisor:monitoria}]}}} \\
        \textbf{Disciplina:}  Teoria e Implementação de Linguagens Computacionais (turma CC)\\
        \textbf{Curso:} Sistemas de Informação\\
        \textbf{Semestre:} 2017.1

\end{enumerate}

%------------------------------------------------------------------------------

\subsubsection{Orienta\c{c}\~{a}o de Trabalhos de Inicia\c{c}\~{a}o Cient\'{\i}fica}
\vspace{0.3cm}

Nada a declarar neste subgrupo.
%\begin{enumerate}
%\renewcommand{\labelenumi}{{\large\bfseries\arabic{enumi}.}}
%
%\item   \textbf{Aluno:} Gorm den Gamle \mbox{\sffamily{\bfseries{[Doc. \ref{advisor:icc}]}}} \\
%        \textbf{Projeto:} Pedras de Jelling \\
%        \textbf{Tema:} Nórdico antigo \\
%        \textbf{Per\'{\i}odo:} 936 - 938\\
%        \textbf{Financiamento:} CAPES
%
%\end{enumerate}

%------------------------------------------------------------------------------

\subsubsection{Orienta\c{c}\~{a}o de Trabalhos de Apoio Acad\^{e}mico}
\vspace{0.3cm}

Nada a declarar neste subgrupo.
%\begin{enumerate}
%\renewcommand{\labelenumi}{{\large\bfseries\arabic{enumi}.}}
%
%\item   \textbf{Aluno:} Qui-Gon \mbox{\sffamily{\bfseries{[Doc. \ref{advisor:bia}]}}}\\
%        \textbf{Projeto:} Uso da Internet das Coisas no acesso multimodal a informações em um Smart Campus.\\
%        \textbf{Tema:} Comunica\c{c}\~{o}es sem Fio \\
%        \textbf{Categoria:} Iniciação acadêmica (Graduando em Ciência da Computação) – Universidade Federal de Pernambuco.\\
%        \textbf{Per\'{\i}odo:} 01 de Outubro de 2014 a 30 de Setembro de 2015 \\
%        \textbf{Financiamento:} Programa Institucional de Bolsa de Incentivo Acadêmico – BIA, Fundação de Amparo à Ciência e Tecnologia do Estado de Pernambuco (FACEPE), processo número: BIA-0155-1.03/14.
%
%\end{enumerate}

%%%%%%%%%%%%%%%%%%%%%%%%%%%%%%%%%%%%%%%%%%%%%%%%%%%%%%%%%%%%%%%%%%%%%%%%%%%%%%%
% Subgrupo 1.2 - Participa\c{c}\~{a}o em Comiss\~{o}es Examinadoras
%%%%%%%%%%%%%%%%%%%%%%%%%%%%%%%%%%%%%%%%%%%%%%%%%%%%%%%%%%%%%%%%%%%%%%%%%%%%%%%
\subsection{Participa\c{c}\~{a}o em Comiss\~{o}es Examinadoras}
\vspace{0.3cm}

\subsubsection{Bancas Examinadoras de Concurso}
\vspace{0.3cm}

\begin{enumerate}
\renewcommand{\labelenumi}{{\large\bfseries\arabic{enumi}.}}
\vspace{0.3cm}

\item       \textbf{Descrição:} Membro da banca examinadora de concurso público para provimento de cargos na carreira de magistério superior do Instituto de Computação da Universidade Federal de Alagoas, na área de estudo \emph{Matemática Aplicada à Computação}. \mbox{\sffamily{\bfseries{[Doc. \ref{app:banca-concurso}]}}}\\%TODO COMPROVANTE BANCA CONCURSO
            \textbf{Instituição:} Instituto de Computação - Universidade Federal de Alagoas\\
            \textbf{Data/Período:} Julho de 2016 %16 de Agosto de 2015

\end{enumerate}

%------------------------------------------------------------------------------

\subsubsection{Bancas Congressos de Inicia\c{c}\~{a}o Cient\'{i}fica ou de Extens\~{a}o}
\vspace{0.3cm}

\begin{enumerate}
\renewcommand{\labelenumi}{{\large\bfseries\arabic{enumi}.}}
\vspace{0.3cm}

\item       \textbf{Descrição:} Participação na Comissão Avaliadora dos trabalhos na área de Ciências Exatas apresentados na 19ª Jornada de Iniciação Científica da FACEPE \mbox{\sffamily{\bfseries{[Doc. \ref{app:2015-jic-facepe}]}}} \\
            \textbf{Instituição:} Fundação de Amparo a Ciência e Tecnologia do Estado de Pernambuco (FACEPE) \\
            \textbf{Data/Período:} 09 a 12 de Junho de 2015

\end{enumerate}

%------------------------------------------------------------------------------

\subsubsection{Bancas de Trabalho de Conclus\~{a}o de Curso}
\vspace{0.3cm}

\begin{enumerate}
\renewcommand{\labelenumi}{{\large\bfseries\arabic{enumi}.}}
\vspace{0.3cm}

\item       \textbf{Aluno:} Filipe Marques Chaves de Arruda \mbox{\sffamily{\bfseries{[Doc. \ref{app:bancas-tcc}]}}}\\
            \textbf{T\'{\i}tulo da Monografia:} Suporte a Interfaces Bidimensionais para Exceções em Java na Plataforma de Desenvolvimento Eclipse.\\
            \textbf{Curso:} Graduação em Ciência da Computação\\
            \textbf{Universidade:} Universidade Federal de Pernambuco (UFPE)\\
            \textbf{Data:} 23 de Fevereiro de 2015\\

\item       \textbf{Aluno:} Diogo Rodrigues Cabral \mbox{\sffamily{\bfseries{[Doc. \ref{app:bancas-tcc}]}}}\\
            \textbf{T\'{\i}tulo da Monografia:} Visualização e Manipulação de Dados em Dispositivos Móveis.\\
            \textbf{Curso:} Graduação em Ciência da Computação\\
            \textbf{Universidade:} Universidade Federal de Pernambuco (UFPE)\\
            \textbf{Data:} 16 de Dezembro de 2015\\

\end{enumerate}

%------------------------------------------------------------------------------

\subsubsection{Bancas de Disserta\c{c}\~{a}o de Mestrado}
\vspace{0.3cm}

\begin{enumerate}
\renewcommand{\labelenumi}{{\large\bfseries\arabic{enumi}.}}
\vspace{0.3cm}


\item       \textbf{Candidato:} Samuel Carlos Romeiro Azevedo Souto \mbox{\sffamily{\bfseries{[Doc. \ref{app:2015-msc-hslb}]}}} \\
            \textbf{T\'{\i}tulo da Disserta\c{c}\~{a}o:} Investigando o Uso e Aplicação de Métricas de Manutenibilidade em Empresas de Software Brasileiras\\
            \textbf{Tipo:} Acadêmico\\
            \textbf{Programa:} Programa de Pós-Graduação em Ciência da Computação\\
            \textbf{Universidade:} Universidade Federal de Pernambuco (UFPE)\\
            \textbf{Data:} 02 de Fevereiro de 2015


\item       \textbf{Candidato:} Diogo Cabral de Almeida \mbox{\sffamily{\bfseries{[Doc. \ref{app:2015-msc-dca-ufal}]}}} \\
            \textbf{T\'{\i}tulo da Disserta\c{c}\~{a}o:} PRIDE: Uma Ferramenta de Detecção de Similaridade em Código-fonte\\
            \textbf{Tipo:} Acadêmico\\
            \textbf{Programa:} Programa de Pós-Graduação em Informática\\
            \textbf{Universidade:} Universidade Federal de Alagoas (UFAL)\\
            \textbf{Data:} 31 de Março de 2015
            
            
\item       \textbf{Candidato:} Jadson José dos Santos \mbox{\sffamily{\bfseries{[Doc. \ref{app:2015-msc-jjs-ufrn}]}}} \\
            \textbf{T\'{\i}tulo da Disserta\c{c}\~{a}o:} Avaliação Sistemática de uma Abordagem para Integração de Funcionalidades em Sistemas Web Clonados\\
            \textbf{Tipo:} Acadêmico\\
            \textbf{Programa:} Programa de Pós-Graduação em Sistemas e Computação\\
            \textbf{Universidade:} Universidade Federal do Rio Grande do Norte (UFRN)\\
            \textbf{Data:} 13 de Agosto de 2015

\item       \textbf{Candidato:} Paulo de Barros e Silva Filho \mbox{\sffamily{\bfseries{[Doc. \ref{app:2015-msc-pbsf-ufpe}]}}} \\
            \textbf{T\'{\i}tulo da Disserta\c{c}\~{a}o:} Static Analysis of Implicit Control Flow: Resolving Java Reflection and Android Intents\\
            \textbf{Tipo:} Acadêmico\\
            \textbf{Programa:} Programa de Pós-Graduação em Ciência da Computação\\
            \textbf{Universidade:} Universidade Federal de Pernambuco (UFPE)\\
            \textbf{Data:} 04 de Março de 2016

\item       \textbf{Candidato:} Larissa Nadja Braz Brasileiro \mbox{\sffamily{\bfseries{[Doc. \ref{app:2015-msc-lnbb-ufcg}]}}} \\
            \textbf{T\'{\i}tulo da Disserta\c{c}\~{a}o:} Uma Técnica para Compilar Sistemas Configuráveis com \#ifdefs baseada no Impacto da Mudança\\
            \textbf{Tipo:} Acadêmico\\
            \textbf{Programa:} Programa de Pós-Graduação em Ciência da Computação\\
            \textbf{Universidade:} Universidade Federal de Campina Grande (UFCG)\\
            \textbf{Data:} 31 de Maio de 2016

\item       \textbf{Candidato:} Heitor Paceli Maranhão \mbox{\sffamily{\bfseries{[Doc. \ref{app:2016-msc-hpm-ufpe}]}}} \\
            \textbf{T\'{\i}tulo da Disserta\c{c}\~{a}o:} Program Synthesis From Denotational Semantics\\
            \textbf{Tipo:} Acadêmico\\
            \textbf{Programa:} Programa de Pós-Graduação em Ciência da Computação\\
            \textbf{Universidade:} Universidade Federal de Pernambuco (UFPE)\\
            \textbf{Data:} 13 de Setembro de 2016

\item       \textbf{Candidato:} Alessandro Borges Rodrigues \mbox{\sffamily{\bfseries{[Doc. \ref{app:2015-mprof-abr-ufpe}]}}} \\
            \textbf{T\'{\i}tulo da Disserta\c{c}\~{a}o:} Uma Abordagem Gradativa de Modernização de Software Monolítico e em Camadas para Serviços\\
            \textbf{Tipo:} Profissional\\
            \textbf{Programa:} Programa de Pós-Graduação em Ciência da Computação\\
            \textbf{Universidade:} Universidade Federal de Pernambuco (UFPE)\\
            \textbf{Data:} 27 de Março de 2017

\item       \textbf{Candidato:} Filipe Marques Chaves de Arruda \mbox{\sffamily{\bfseries{[Doc. \ref{app:2015-msc-fmca-ufpe}]}}} \\
            \textbf{T\'{\i}tulo da Disserta\c{c}\~{a}o:} Test Automation from Natural Language with Reusable Capture \& Replay and Consistency Analysis\\
            \textbf{Tipo:} Acadêmico\\
            \textbf{Programa:} Programa de Pós-Graduação em Ciência da Computação\\
            \textbf{Universidade:} Universidade Federal de Pernambuco (UFPE)\\
            \textbf{Data:} 31 de Março de 2017
\end{enumerate}

%------------------------------------------------------------------------------

\subsubsection{Bancas de Qualifica\c{c}\~ao / Proposta de Tese de Doutorado}
\vspace{0.3cm}

\begin{enumerate}
\renewcommand{\labelenumi}{{\large\bfseries\arabic{enumi}.}}
\vspace{0.3cm}

\item       \textbf{Candidato:} Melina Mongiovi \mbox{\sffamily{\bfseries{[Doc. \ref{app:2015-quali-phd-mm-ufcg}]}}} \\
            \textbf{T\'{\i}tulo da Qualifica\c{c}\~{a}o:} Scaling Testing of Refactoring Engines\\
            \textbf{Programa:} Programa de Pós-Graduação em Ciência da Computação\\
            \textbf{Universidade:} Universidade Federal de Campina Grande (UFCG)\\
            \textbf{Data:} 08 de Outubro de 2015

\item       \textbf{Candidato:} Gabriela Guedes de Souza \mbox{\sffamily{\bfseries{[Doc. \ref{app:2016-quali-phd-ggs-ufpe}]}}} \\
            \textbf{T\'{\i}tulo da Qualifica\c{c}\~{a}o:} Modelling, Configuring, and Evolving Requirements of Dynamic Software Product Lines\\
            \textbf{Programa:} Programa de Pós-Graduação em Ciência da Computação\\
            \textbf{Universidade:} Universidade Federal de Pernambuco (UFPE)\\
            \textbf{Data:} 20 de Abril de 2016

\item       \textbf{Candidato:} Felipe Ebert \mbox{\sffamily{\bfseries{[Doc. \ref{app:2016-quali-phd-fe-ufpe}]}}} \\
            \textbf{T\'{\i}tulo da Qualifica\c{c}\~{a}o:} Using Task Context to Assist Code Review\\
            \textbf{Programa:} Programa de Pós-Graduação em Ciência da Computação\\
            \textbf{Universidade:} Universidade Federal de Pernambuco (UFPE)\\
            \textbf{Data:} 02 de Setembro de 2016

\item       \textbf{Candidato:} Filipe Rafael Gomes Varjão \mbox{\sffamily{\bfseries{[Doc. \ref{app:2016-quali-phd-frgv-ufpe}]}}} \\
            \textbf{T\'{\i}tulo da Qualifica\c{c}\~{a}o:} Tradução em Alto Nível de Objetos Java para Processos em Erlang\\
            \textbf{Programa:} Programa de Pós-Graduação em Ciência da Computação\\
            \textbf{Universidade:} Universidade Federal de Pernambuco (UFPE)\\
            \textbf{Data:} 06 de Setembro de 2016

\end{enumerate}

%------------------------------------------------------------------------------

\subsubsection{Bancas de Tese de Doutorado}
%\vspace{0.3cm}

\begin{enumerate}
\renewcommand{\labelenumi}{{\large\bfseries\arabic{enumi}.}}
\vspace{0.3cm}

\item       \textbf{Candidato:} Lucas Albertins de Lima \mbox{\sffamily{\bfseries{[Doc. \ref{app:2016-phd-lal-ufpe}]}}} \\
            \textbf{T\'{\i}tulo da Tese:} Formalisation of SysML Design Models and an Analysis Strategy using Refinement\\
            \textbf{Programa:} Programa de Pós-Graduação em Ciência da Computação\\
            \textbf{Universidade:} Universidade Federal de Pernambuco (UFPE)\\
            \textbf{Data:} 03 de Março de 2016

\item       \textbf{Candidato:} Melina Mongiovi \mbox{\sffamily{\bfseries{[Doc. \ref{app:2016-phd-mm-ufcg}]}}} \\
            \textbf{T\'{\i}tulo da Tese:} Scaling Testing of Refactoring Engines\\
            \textbf{Programa:} Programa de Pós-Graduação em Ciência da Computação\\
            \textbf{Universidade:} Universidade Federal de Campina Grande (UFCG)\\
            \textbf{Data:} 29 de Novembro de 2016

\item       \textbf{Candidato:} Tarciana Dias da Silva \mbox{\sffamily{\bfseries{[Doc. \ref{app:2016-phd-tds-ufpe}]}}} \\
            \textbf{T\'{\i}tulo da Tese:} Validating Transformations of Programs using the Alloy Analyzer\\
            \textbf{Programa:} Programa de Pós-Graduação em Ciência da Computação\\
            \textbf{Universidade:} Universidade Federal de Pernambuco (UFPE)\\
            \textbf{Data:} 23 de Agosto de 2017

%\item       \textbf{Candidato:} Gabriela Guedes de Souza \mbox{\sffamily{\bfseries{[Doc. \ref{app:2017-phd-ggs-ufpe}]}}} \\
%            \textbf{T\'{\i}tulo da Tese:} Contextual Goal Models for Dynamic Software Product Lines\\
%            \textbf{Programa:} Programa de Pós-Graduação em Ciência da Computação\\
%            \textbf{Universidade:} Universidade Federal de Pernambuco (UFPE)\\
%            \textbf{Data:} 14 de Setembro de 2017
%
\end{enumerate}

%------------------------------------------------------------------------------

\subsubsection{Participa\c{c}\~{a}o em Bancas de Sele\c{c}\~{a}o para ingresso e exames de Qualifica\c{c}\~ao de Programa de P\'{o}-Gradua\c{c}\~{a}o \textit{Stricto Sensu}}

Nada a declarar neste subgrupo.
%\begin{enumerate}
%\renewcommand{\labelenumi}{{\large\bfseries\arabic{enumi}.}}
%\vspace{0.3cm}
%
%\item       \textbf{Descrição:} Membro Externo da Comissão de Seleção para o Doutorado Sanduíche - PDSE da CAPES \mbox{\sffamily{\bfseries{[Doc. \ref{app:2015-pgcomp-pdse}]}}} \\
%            \textbf{Programa:} Programa de Pós-Graduação em Ciência da Computação (PGCOMP) do aluno Alcemir Rodrigues Santos\\
%            \textbf{Instituição:} Universidade Federal da Bahia (UFBA) \\
%            \textbf{Per\'{\i}odo de Realiza\c{c}\~{a}o:} 15 de Junho de 2015
%
%\end{enumerate}

%%%%%%%%%%%%%%%%%%%%%%%%%%%%%%%%%%%%%%%%%%%%%%%%%%%%%%%%%%%%%%%%%%%%%%%%%%%%%%%
% Subgrupo 1.3 - Atividades de Ensino na Graduação e na Pós-Graduação
%%%%%%%%%%%%%%%%%%%%%%%%%%%%%%%%%%%%%%%%%%%%%%%%%%%%%%%%%%%%%%%%%%%%%%%%%%%%%%%

\subsection{Atividades de Ensino na Graduação e na Pós-Graduação}
\vspace{0.3cm}

\subsubsection{Atividades de Ensino: Disciplinas na Gradua\c{c}\~{a}o}
\vspace{0.3cm}

Na Tabela \ref{Tab:Disc_Grad}, estão listadas as disciplinas ministradas nos cursos de gradua\c{c}\~{a}o no per\'{\i}odo de 25/08/2014 a 25/08/2017. As disciplinas est\~{a}o organizadas por semestre, com a identifica\c{c}\~{a}o do curso, da turma e da disciplina, esta \'{u}ltima com sua respectiva carga hor\'{a}ria. Adicionalmente, consta na referida Tabela a carga hor\'{a}ria m\'{e}dia semestral ao longo do per\'{\i}odo de avalia\c{c}\~{a}o.

\begin{table}[!htpb]
\centering \small
\caption{\texttt{Disciplinas de gradua\c{c}\~{a}o lecionadas no per\'{\i}odo considerado} }
\begin{tabular}{ccccc}
\toprule
\textbf{Semestre} & \textbf{Curso} & \textbf{Turma} & \textbf{Disciplina} & \textbf{C.H.} \\
\otoprule
$2014.2$ & Sistemas de Informação & S2 & IF969 - Algoritmos e Estruturas de Dados & 02\\
  & Sistemas de Informação & S0 & IF1001 - Programação 3 & 02\\
\cmidrule{1-5}
\multicolumn{4}{r}{\textbf{Carga Hor\'{a}ria M\'{e}dia Semestral} (2014)} & \textbf{04} \\
\otoprule
$2015.1$ & Sistemas de Informação & S2 & IF969 - Algoritmos e Estruturas de Dados & 02\\
  & Sistemas de Informação & S0 & IF1001 - Programação 3 & 02\\
\cmidrule{1-5}
$2015.2$ & Sistemas de Informação & S2 & IF969 - Algoritmos e Estruturas de Dados & 02\\
  & Ciência da Computação & I5 & IF688 - Teoria e Impl. de Ling. Computacionais & 02\\
\cmidrule{1-5}
\multicolumn{4}{r}{\textbf{Carga Hor\'{a}ria M\'{e}dia Semestral} (2015)} & \textbf{04} \\
\otoprule
  & Ciência da Computação & I5 & IF688 - Teoria e Impl. de Ling. Computacionais & 02\\
$2016.1$ & Sistemas de Informação & S2 & IF969 - Algoritmos e Estruturas de Dados & 02\\
  & Engenharia da Computação & E7 & IF688 - Teoria e Impl. de Ling. Computacionais & 02\\
\cmidrule{1-5}
$2016.2$ & Ciência da Computação & I5 & IF688 - Teoria e Impl. de Ling. Computacionais & 02\\
\cmidrule{1-5}
\multicolumn{4}{r}{\textbf{Carga Hor\'{a}ria M\'{e}dia Semestral} (2016)} & \textbf{04} \\
\otoprule
$2017.1$ & Sistemas de Informação & S0 & IF1001 - Programação 3 & 02\\
  & Ciência da Computação & I5 & IF688 - Teoria e Impl. de Ling. Computacionais & 02\\
\cmidrule{1-5}
$2017.2$ & Ciência da Computação & I5 & IF688 - Teoria e Impl. de Ling. Computacionais & 02\\
  & Ciência da Computação & I9 & IF710 - Programação com Componentes & 02\\
\cmidrule{1-5}
\multicolumn{4}{r}{\textbf{Carga Hor\'{a}ria M\'{e}dia Semestral} (2017)} & \textbf{04} \\
\bottomrule
\end{tabular}
\label{Tab:Disc_Grad}
\end{table}

%%%%%%%%%%%%%%%%%%%%%%%%%%%%%%%%%%%%%%%%%%%%%%%%%%%%%%%%%%%%%%%%%%%%%%%%%%%%%%%
% Subgrupo 1.4 - Avaliação Didática Docente pelo Discente
%%%%%%%%%%%%%%%%%%%%%%%%%%%%%%%%%%%%%%%%%%%%%%%%%%%%%%%%%%%%%%%%%%%%%%%%%%%%%%%
%\vspace{5.0cm}
\subsection{Avaliação Didática Docente pelo Discente}
\vspace{0.3cm}

%Nada a declarar neste subgrupo.

\begin{enumerate}
\renewcommand{\labelenumi}{{\large\bfseries\arabic{enumi}.}}

\item   \textbf{Período:} 2014.2 \\
        \textbf{Disciplina:} IF1001 - Programação 3, Turma: S0\\
        \textbf{Disciplina:} IF969 - Algoritmos e Estruturas de Dados, Turma: S2\\
        \textbf{Não houve avaliação didática docente pelo discente}.

\item   \textbf{Período:} 2015.1 \\
        \textbf{Disciplina:} IF1001 - Programação 3, Turma: S0\\
        \textbf{Qtde. Alunos:} 41\\
        \textbf{Adesão:} 27 ($65.85\%$)\\
        \textbf{Mediana das Afirmativas:} 6 \textbf{[Doc. \ref{evaluation:2015-1-if1001}]}

\item   \textbf{Período:} 2015.1 \\
        \textbf{Disciplina:} IF969 - Algoritmos e Estruturas de Dados, Turma: S2\\
        \textbf{Qtde. Alunos:} 41\\
        \textbf{Adesão:} 19 ($90.48\%$)\\
        \textbf{Mediana das Afirmativas:} 6 \textbf{[Doc. \ref{evaluation:2015-1-if969}]}

\item   \textbf{Período:} 2015.2 \\
        \textbf{Disciplina:} IF969 - Algoritmos e Estruturas de Dados, Turma: S2\\
        \textbf{Qtde. Alunos:} 39\\
        \textbf{Adesão:} 32 ($82.05\%$)\\
        \textbf{Mediana das Afirmativas:} 6 \textbf{[Doc. \ref{evaluation:2015-2-if969}]}

\item   \textbf{Período:} 2015.2 \\
        \textbf{Disciplina:} IF688 - Teoria e Implementação de Linguagens Computacionais, Turma: I5\\
        \textbf{Qtde. Alunos:} 34\\
        \textbf{Adesão:} 22 ($64.71\%$)\\
        \textbf{Mediana das Afirmativas:} 6 \textbf{[Doc. \ref{evaluation:2015-2-if688}]}

\item   \textbf{Período:} 2016.1 \\
        \textbf{Disciplina:} IF969 - Algoritmos e Estruturas de Dados, Turma: S2\\
        \textbf{Qtde. Alunos:} 20\\
        \textbf{Adesão:} 13 ($65\%$)\\
        \textbf{Mediana das Afirmativas:} 6 \textbf{[Doc. \ref{evaluation:2016-1-if969}]}

\item   \textbf{Período:} 2016.1 \\
        \textbf{Disciplina:} IF688 - Teoria e Implementação de Linguagens Computacionais, Turma: I5\\
        \textbf{Qtde. Alunos:} 27\\
        \textbf{Adesão:} 16 ($59.26\%$)\\
        \textbf{Mediana das Afirmativas:} 6 \textbf{[Doc. \ref{evaluation:2016-1-if688}]}

\item   \textbf{Período:} 2016.1 \\
        \textbf{Disciplina:} IF688 - Teoria e Implementação de Linguagens Computacionais, Turma: E7\\
        \textbf{Qtde. Alunos:} 25\\
        \textbf{Adesão:} 9 ($36\%$)\\
        \textbf{Mediana das Afirmativas:} 6 \textbf{[Doc. \ref{evaluation:2016-1-if688-ec}]}

\item   \textbf{Período:} 2016.2 \\
        \textbf{Disciplina:} IF688 - Teoria e Implementação de Linguagens Computacionais, Turma: I5\\
        \textbf{Qtde. Alunos:} 28\\
        \textbf{Adesão:} 15 ($53.6\%$)\\
        \textbf{Mediana das Afirmativas:} 6 \textbf{[Doc. \ref{evaluation:2016-2-if688}]}

\item   \textbf{Período:} 2017.1 \\
        \textbf{Disciplina:} IF688 - Teoria e Implementação de Linguagens Computacionais, Turma: I5\\
        \textbf{Qtde. Alunos:} 47\\
        \textbf{Adesão:} 17 ($36.2\%$)\\
        \textbf{Mediana das Afirmativas:} 6 \textbf{[Doc. \ref{evaluation:2017-1-if688}]}

\item   \textbf{Período:} 2017.1 \\
        \textbf{Disciplina:} IF1001 - Programação 3, Turma: S0\\
        \textbf{Qtde. Alunos:} 50\\
        \textbf{Adesão:} Menos de 30\% do total de discentes da turma respondeu a avaliação.


\end{enumerate}

%%%%%%%%%%%%%%%%%%%%%%%%%%%%%%%%%%%%%%%%%%%%%%%%%%%%%%%%%%%%%%%%%%%%%%%%%%%%%%%
% Grupo 2: Atividades de Produ\c{c}\~{a}o Cient\'{\i}fica e T\'{e}cnica, Art\'{i}stica e Cultural
%%%%%%%%%%%%%%%%%%%%%%%%%%%%%%%%%%%%%%%%%%%%%%%%%%%%%%%%%%%%%%%%%%%%%%%%%%%%%%%
\newpage
\section{Atividades de Produ\c{c}\~{a}o Cient\'{\i}fica e T\'{e}cnica, Art\'{i}stica e Cultural}

%%%%%%%%%%%%%%%%%%%%%%%%%%%%%%%%%%%%%%%%%%%%%%%%%%%%%%%%%%%%%%%%%%%%%%%%%%%%%%%
% Subgrupo 2.1 - Produtividade de Pesquisa
%%%%%%%%%%%%%%%%%%%%%%%%%%%%%%%%%%%%%%%%%%%%%%%%%%%%%%%%%%%%%%%%%%%%%%%%%%%%%%%
\subsection{Produtividade de Pesquisa}
\vspace{0.3cm}

Nada a declarar neste subgrupo.
%%------------------------------------------------------------------------------
%
%\subsubsection{Bolsista de produtividade em pesquisa e em inova\c{c}\~{a}o tecnol\'{o}gica}
%\vspace{0.3cm}
%
%\begin{enumerate}
%\renewcommand{\labelenumi}{{\large\bfseries\arabic{enumi}.}}
%
%\item \textbf{T\'{\i}tulo do projeto:} The Death Star \textbf{[Doc. \ref{app:bolsista-prod}]}\\
%      \textbf{N\'{u}mero do processo:} DV-1123-5.8/13\\
%      \textbf{Financiador/Edital:} Conselho Nacional de Desenvolvimento Científico e Tecnológico, Edital MCTI/CNPq Nº 14/2013\\
%      \textbf{Per\'{\i}odo (in\'{\i}cio-fim):} 2015 - 2018\\
%      \textbf{Situação:} Desenvolvimento tecnológico\\
%      \textbf{Natureza:} Pesquisa.
%
%\end{enumerate}

%------------------------------------------------------------------------------

\subsubsection{Participa\c{c}\~{a}o em Eventos Cient\'{\i}ficos (com apresenta\c{c}\~{a}o de trabalho ou oferecimento de cursos, palestras ou debates)}
\vspace{0.3cm}

\begin{enumerate}
\renewcommand{\labelenumi}{{\large\bfseries\arabic{enumi}.}}

\item   \textbf{Evento:} V Congresso Brasileiro de Software: Teoria e Prática \mbox{\sffamily{\bfseries{[Doc. \ref{event:2014-cbsoft}]}}} \\
        \textbf{Propósito:} Participante\\
        \textbf{Período:} 28 de Setembro a 03 de Outubro de 2014\\
        \textbf{Local:} Maceió-AL, Brasil.

\item   \textbf{Evento:} Feature-oriented Software Development Meeting (FOSD) \mbox{\sffamily{\bfseries{[Doc. \ref{event:2015-fosd}]}}} \\
        \textbf{Propósito:} Apresentação de Trabalho\\
        \textbf{Período:} 13 a 16 de Maio de 2015\\
        \textbf{Local:} International Academy Traunkirchen, Áustria.

\item   \textbf{Evento:} 37th International Conference on Software Engineering \mbox{\sffamily{\bfseries{[Doc. \ref{event:2015-icse}]}}} \\
        \textbf{Propósito:} Participante\\
        \textbf{Período:} 18 a 23 de Maio de 2015\\
        \textbf{Local:} Florença, Itália.

\item   \textbf{Evento:} 19th International Software Product Line Conference
 \mbox{\sffamily{\bfseries{[Doc. \ref{event:2015-splc}]}}} \\
        \textbf{Propósito:} Apresentação de Trabalho\\
        \textbf{Período:} 20 a 24 de Julho de 2015\\
        \textbf{Local:} Nashville-TN, Estados Unidos.

\item   \textbf{Evento:} VI Congresso Brasileiro de Software: Teoria e Prática \mbox{\sffamily{\bfseries{[Doc. \ref{event:2015-cbsoft}]}}} \\
        \textbf{Propósito:} Participante\\
        \textbf{Período:} 21 a 25 de Setembro de 2015\\
        \textbf{Local:} Belo Horizonte-MG, Brasil.

\item   \textbf{Evento:} ACM SIGPLAN conference on Systems, Programming, Languages and Applications: Software for Humanity (SPLASH) \mbox{\sffamily{\bfseries{[Doc. \ref{event:2016-splash}]}}} \\
        \textbf{Propósito:} Participante - Apresentação de Trabalho e Organização de Workshop\\
        \textbf{Período:} 30 de Outubro a 04 de Novembro de 2016\\
        \textbf{Local:} Amsterdã, Holanda.

\item   \textbf{Evento:} DeepSpec Summer School on Verified Systems \mbox{\sffamily{\bfseries{[Doc. \ref{event:2017-deepspec}]}}} \\
        \textbf{Propósito:} Participante\\
        \textbf{Período:} 13 a 28 de Julho de 2017\\
        \textbf{Local:} Filadélfia, Estados Unidos.

\end{enumerate}

%------------------------------------------------------------------------------
% TODO
\subsubsection{Autoria de artigos completos publicados em anais de congresso, em jornais e revistas de circulação nacional e internacional na sua área}
\vspace{0.3cm}

\begin{enumerate}
\renewcommand{\labelenumi}{{\large\bfseries\arabic{enumi}.}}

\item \textbf{TEIXEIRA, Leopoldo}; ALVES, Vander ; BORBA, Paulo ; GHEYI, Rohit. \textbf{A Product Line of Theories for Reasoning about Safe Evolution of Product Lines}. In: \emph{The 19th International Software Product Line Conference: New Directions in Systems and Software Product Line Engineering}, 2015, Nashville, TN. \textbf{[Doc. \ref{conf:2015-splc-1}]}
%
\item \textbf{TEIXEIRA, Leopoldo}; BORBA, Paulo ; GHEYI, Rohit. \textbf{Safe Evolution of Product Populations and Multi Product Lines}. Em: \emph{The 19th International Software Product Line Conference: New Directions in Systems and Software Product Line Engineering}, 2015, Nashville, TN. \textbf{[Doc. \ref{conf:2015-splc-2}]}
%
\item MEDEIROS, Flávio; RODRIGUES, Iran; RIBEIRO, Márcio; \textbf{TEIXEIRA, Leopoldo};  GHEYI, Rohit. \textbf{An Empirical Study on Configuration-Related Type Issues}. Em: \emph{14th International Conference on Generative Programming: Concepts \& Experience (GPCE'15)}, 2015, Pittsburgh, PA. \textbf{[Doc. \ref{conf:2015-gpce}]}

\item BENBASSAT, Fernando ; BORBA, Paulo; \textbf{TEIXEIRA, Leopoldo}. \textbf{Safe Evolution of Software Product Lines: Feature Extraction Scenarios}. Em: \emph{X Simpósio Brasileiro de Componentes, Arquiteturas e Reutilização de Software (SBCARS)}, 2016, Maringá, PR. \textbf{[Doc. \ref{conf:2016-sbcars}]}

\item SAMPAIO, Gabriela ; BORBA, Paulo; \textbf{TEIXEIRA, Leopoldo}. \textbf{Partially Safe Evolution of Software Product Lines}. Em: \emph{20th International Systems and Software Product Line Conference}, 2016, Beijing, China. \textbf{[Doc. \ref{conf:2016-splc}]}

\item BRAZ, Larissa; GHEYI, Rohit; MONGIOVI, Melina; RIBEIRO, Márcio; MEDEIROS, Flávio; \textbf{TEIXEIRA, Leopoldo}. \textbf{A Change-Centric Approach to Compile Configurable Systems with \#ifdefs}. Em: \emph{International Conference on Generative Programming: Concepts \& Experiences (GPCE'2016)}, 2016, Amsterdã, Holanda. \textbf{[Doc. \ref{conf:2016-gpce}]}

\end{enumerate}

%------------------------------------------------------------------------------
% TODO
\subsubsection{Arbitragem de Artigos Técnico-Científicos Nacionais e Internacionais na sua área de atuação}
\vspace{0.3cm}

\begin{enumerate}
\renewcommand{\labelenumi}{{\large\bfseries\arabic{enumi}.}}

\item   \textbf{Peri\'{o}dico:} Journal of Systems and Software \textbf{[Doc. \ref{reviewer:2014-jss}]}\\
        \textbf{Editora:} Elsevier\\
        \textbf{ISSN:} 0164-1212\\
        \textbf{URL:} \url{https://www.journals.elsevier.com/journal-of-systems-and-software/}

\item   \textbf{Peri\'{o}dico:} Software: Practice and Experience \textbf{[Doc. \ref{reviewer:2015-spe}]}\\
        \textbf{Editora:} Wiley\\
        \textbf{ISSN:} 1097-024X\\
        \textbf{URL:} \url{http://onlinelibrary.wiley.com/journal/10.1002/(ISSN)1097-024X}

\item   \textbf{Peri\'{o}dico:} ACM Transactions on Software Engineering and Methodology \textbf{[Doc. \ref{reviewer:2015-tosem}]}\\
        \textbf{Editora:} ACM\\
        \textbf{ISSN:} 1049-331X\\
        \textbf{URL:} \url{http://tosem.acm.org/}

\item   \textbf{Peri\'{o}dico:} Journal of Software Engineering Research and Development \textbf{[Doc. \ref{reviewer:2015-jserd}]}\\
        \textbf{Editora:} SpringerOpen\\
        \textbf{ISSN:} 2195-1721\\
        \textbf{URL:} \url{https://jserd.springeropen.com/}

\item   \textbf{Peri\'{o}dico:} Journal of Computer Science and Technology \textbf{[Doc. \ref{reviewer:2015-jcst}]}\\
        \textbf{Editora:} Springer\\
        \textbf{ISSN:} 1000-9000\\
        \textbf{URL:} \url{https://jserd.springeropen.com/}

\item   \textbf{Peri\'{o}dico:} IEEE Transactions on Software Engineering \textbf{[Doc. \ref{reviewer:2016-tse}]}\\
        \textbf{Editora:} IEEE\\
        \textbf{ISSN:} 0098-5589\\
        \textbf{URL:} \url{https://www.computer.org/csdl/journal/ts}

\item   \textbf{Comitê de Programa:} Latin-American School on Software Engineering (ELA-ES) 2015. \textbf{[Doc. \ref{reviewer:2015-ela-es}]}

\item   \textbf{Comitê de Programa:} Simpósio Brasileiro de Componentes, Arquiteturas e Reuso de Software (SBCARS) 2014---2017. \textbf{[Doc. \ref{reviewer:2014-2017-sbcars}]}

\item   \textbf{Comitê de Programa:} Simpósio Brasileiro de Métodos Formais (SBMF) 2015. \textbf{[Doc. \ref{reviewer:2015-sbmf}]}

\item   \textbf{Comitê de Programa:} Simpósio Brasileiro de Linguagens de Programação (SBLP) 2016---2017. \textbf{[Doc. \ref{reviewer:2016-2017-sblp}]}

\item   \textbf{Comitê de Programa:} Simpósio Brasileiro de Engenharia de Software (SBES) 2016. \textbf{[Doc. \ref{reviewer:2016-sbes}]}

\item   \textbf{Comitê de Programa:} \emph{Intel® Embedded Systems Competition 2016}. \textbf{[Doc. \ref{reviewer:2016-intel}]}

\item   \textbf{Comitê de Programa:} \emph{International Workshop on Formal Methods and Analysis in Software Product Line Engineering (FMSPLE'2016)}. \textbf{[Doc. \ref{reviewer:2016-fmsple}]}

\item   \textbf{Comitê de avaliação de artefatos:} \emph{European Conference on Object-Oriented Programming (ECOOP'2016 AEC)}. \textbf{[Doc. \ref{reviewer:2016-ecoop-aec}]}

\item   \textbf{Comitê de Programa:} Workshop de Teses e Dissertações do CBSoft (WTDSoft 2017). \textbf{[Doc. \ref{reviewer:2017-wtdsoft}]}

\item   \textbf{Comitê de Programa:} \emph{Software Product Lines and Software Ecosystems track (SPLSeco @ SEAA 2017)}. \textbf{[Doc. \ref{reviewer:2017-splseco}]}

\end{enumerate}

%------------------------------------------------------------------------------
% TODO
\subsubsection{Coordena\c{c}\~{a}o e/ou Participa\c{c}\~{a}o em Projetos Aprovados por \'{O}rg\~{a}os de Fomento}
\vspace{0.3cm}

\begin{enumerate}
\renewcommand{\labelenumi}{{\large\bfseries\arabic{enumi}.}}

\item \textbf{T\'{\i}tulo do projeto:} Ferramentas para Evolução Segura de Linhas de Produtos de Software \textbf{[Doc. \ref{project:2015-facepe-ppp}]}\\
      \textbf{Fun\c{c}\~{a}o no projeto:} Coordenador\\
      \textbf{N\'{u}mero do processo:} APQ-0570-1.03/14\\
      \textbf{Financiador/Edital:} Fundação de Amparo à Ciência e Tecnologia do Estado de Pernambuco (FACEPE), Edital 09/2014 - Programa de Infraestrutura para Jovens Pesquisadores (Programa Primeiros Projetos – PPP/Facepe/CNPq)\\
      \textbf{Per\'{\i}odo (in\'{\i}cio-fim):} 2015 - 2020 (pendente de recebimento dos recursos)\\
      \textbf{Natureza:} Pesquisa.

\item \textbf{T\'{\i}tulo do projeto:} Engenharia de Software para Cidades Inteligentes (ESCIn) \textbf{[Doc. \ref{project:2015-facepe-pronex}]}\\
      \textbf{Fun\c{c}\~{a}o no projeto:} Integrante\\
      \textbf{N\'{u}mero do processo:} APQ 0388-1.03/14\\
      \textbf{Financiador/Edital:} Fundação de Amparo à Ciência e Tecnologia do Estado de Pernambuco (FACEPE), Edital 10/2014 - Programa de Apoio a Núcleos de Excelência (Pronex/Facepe/CNPq)\\
      \textbf{Per\'{\i}odo (in\'{\i}cio-fim):} 2015 - 2019\\
      \textbf{Natureza:} Pesquisa.

\item \textbf{T\'{\i}tulo do projeto:} Suporte Ferramental à Evolução de Linhas de Produtos de Software \textbf{[Doc. \ref{project:2017-universal-cnpq}]}\\
      \textbf{Fun\c{c}\~{a}o no projeto:} Coordenador\\
      \textbf{N\'{u}mero do processo:} 409335/2016-9\\
      \textbf{Financiador/Edital:} Conselho Nacional de Desenvolvimento Científico e Tecnológico (CNPq), Chamada: Universal 01/2016 - Faixa A\\
      \textbf{Per\'{\i}odo (in\'{\i}cio-fim):} 2017 - 2020\\
      \textbf{Natureza:} Pesquisa.

\item \textbf{T\'{\i}tulo do projeto:} Combinando Gamificação e Redes Sociais para Melhorar a Prevenção e Controle da Zika \textbf{[Doc. \ref{project:2017-newton-zika}]}\\
      \textbf{Fun\c{c}\~{a}o no projeto:} Integrante\\
      \textbf{N\'{u}mero do processo:} 60030 1201/2016\\
      \textbf{Financiador/Edital:} Fundação de Amparo à Pesquisa do Estado de Alagoas (FAPEAL), EDITAL FAPEAL/British Council – Chamada Institutional Links – Vírus Zika\\
      \textbf{Per\'{\i}odo (in\'{\i}cio-fim):} 2017 - 2019\\
      \textbf{Natureza:} Pesquisa.

\end{enumerate}

%------------------------------------------------------------------------------

\subsubsection{Consultoria \`{a}s Institui\c{c}\~{o}es de Fomento \`{a} Pesquisa, Ensino e Extens\~{a}o}
\vspace{0.3cm}

Nada a declarar neste subgrupo.

% \begin{enumerate}
% \renewcommand{\labelenumi}{{\large\bfseries\arabic{enumi}.}}

% \item   \textbf{Fun\c{c}\~{a}o:} Avaliador de Projetos de Pesquisa (modalidade Subven\c{c}\~{a}o Econ\^{o}mica - PAPPE Integra\c{c}\~{a}o) - Edital 10.2/2012 \textbf{[Doc. \ref{consulting:2015-facepe-pepe}]}\\
%         \textbf{Institui\c{c}\~{a}o:} Fundação de Amparo à Ciência e Tecnologia do Estado de Pernambuco (FACEPE).

%%%%%%%
%AVALIADOR PIBIC UFRN
%%%%%%%

%\end{enumerate}

%------------------------------------------------------------------------------

\subsubsection{Pr\^{e}mios Recebidos pela Produ\c{c}\~{a}o Cient\'{\i}fica e T\'{e}cnica}
\vspace{0.3cm}

Nada a declarar neste subgrupo.

%\begin{enumerate}
%\renewcommand{\labelenumi}{{\large\bfseries\arabic{enumi}.}}
%
%\item Melhor artigo publicado no Simpósio Intergálito da República, 2014 \textbf{[Doc. \ref{award:2014}]}.
%
%\end{enumerate}

%%%%%%%%%%%%%%%%%%%%%%%%%%%%%%%%%%%%%%%%%%%%%%%%%%%%%%%%%%%%%%%%%%%%%%%%%%%%%%%
% Subgrupo 2.2 - Produção Científica
%%%%%%%%%%%%%%%%%%%%%%%%%%%%%%%%%%%%%%%%%%%%%%%%%%%%%%%%%%%%%%%%%%%%%%%%%%%%%%%

\subsection{Produção Científica}
\vspace{0.3cm}

%------------------------------------------------------------------------------

\subsubsection{Trabalhos Publicados em Peri\'{o}dicos Especializados do Pa\'{\i}s ou do Exterior}
\vspace{0.3cm}

\begin{enumerate}
\renewcommand{\labelenumi}{{\large\bfseries\arabic{enumi}.}}

\item ALFÉREZ, Mauricio; BONIFÁCIO, Rodrigo; \textbf{TEIXEIRA, Leopoldo}; ACCIOLY, Paola; KULESZA, Uirá; MOREIRA, Ana; ARAÚJO, João; BORBA, Paulo. \textbf{Evaluating scenario-based SPL requirements approaches: the case for modularity, stability and expressiveness}. \emph{Requirements Engineering}, v. 19, p. 355-376, 2014. \textbf{[Doc. \ref{journal:2014-re}]}

\item MONGIOVI, Melina; GHEYI, Rohit; Soares, Gustavo; \textbf{TEIXEIRA, Leopoldo}; BORBA, Paulo. \textbf{Making refactoring safer through impact analysis}. \emph{Science of Computer Programming (Print)}, v. 93, p. 39-64, 2014. \textbf{[Doc. \ref{journal:2014-scp}]}

\item NEVES, Laís; BORBA, Paulo; ALVES, Vander; TURNES, Lucineia; \textbf{TEIXEIRA, Leopoldo}; SENA, Demóstenes; KULESZA, Uirá. \textbf{Safe Evolution Templates for Software Product Lines}. \emph{Journal of Systems and Software}, v. 106, p. 42-58, 2015. \textbf{[Doc. \ref{journal:2015-jss}]}

\item PASSOS, Leonardo; \textbf{TEIXEIRA, Leopoldo}; DINTZNER, Nicolas; APEL, Sven; WASOWSKI, Andrzej; CZARNECKI, Krzysztof; BORBA, Paulo; GUO, Jianmei. \textbf{Coevolution of variability models and related software artifacts}. \emph{Empirical Software Engineering}, v. 21, p. 1744-1793, 2016. \textbf{[Doc. \ref{journal:2016-ese}]}
%

\item MEDEIROS, Flavio; RODRIGUES, Iran; RIBEIRO, Márcio; \textbf{TEIXEIRA, Leopoldo}; GHEYI, Rohit. \textbf{An empirical study on configuration-related issues: investigating undeclared and unused identifiers}. \emph{ACM SIGPLAN NOTICES}, v. 51, p. 35-44, 2016. \textbf{[Doc. \ref{journal:2016-acm-sigplan-notices}]}

\item BRAZ, Larissa; GHEYI, Rohit; MONGIOVI, Melina; RIBEIRO, Márcio; MEDEIROS, Flávio; \textbf{TEIXEIRA, Leopoldo}. \textbf{A Change-Centric Approach to Compile Configurable Systems with \#ifdefs}. \emph{ACM SIGPLAN NOTICES}, v. 52, p. 109-119, 2017. \textbf{[Doc. \ref{journal:2017-acm-sigplan-notices}]}

\end{enumerate}

%%%%%%%%%%%%%%%%%%%%%%%%%%%%%%%%%%%%%%%%%%%%%%%%%%%%%%%%%%%%%%%%%%%%%%%%%%%%%%%
% Grupo 3: Atividades de Extens\~{a}o
%%%%%%%%%%%%%%%%%%%%%%%%%%%%%%%%%%%%%%%%%%%%%%%%%%%%%%%%%%%%%%%%%%%%%%%%%%%%%%%
\newpage
\section{Atividades de Extens\~{a}o}

%%%%%%%%%%%%%%%%%%%%%%%%%%%%%%%%%%%%%%%%%%%%%%%%%%%%%%%%%%%%%%%%%%%%%%%%%%%%%%%
% Subgrupo 3.1 - Coordenação e Orientação
%%%%%%%%%%%%%%%%%%%%%%%%%%%%%%%%%%%%%%%%%%%%%%%%%%%%%%%%%%%%%%%%%%%%%%%%%%%%%%%
\subsection{Subgrupo 3.1 - Coordenação e Orientação}
\vspace{0.3cm}

Nada a declarar neste subgrupo.

%\begin{enumerate}
%\renewcommand{\labelenumi}{{\large\bfseries\arabic{enumi}.}}
%
%    \item Coordenação de Projeto de Extensão \textbf{[Doc. \ref{extensao:2015}]}
%
%\end{enumerate}

%%%%%%%%%%%%%%%%%%%%%%%%%%%%%%%%%%%%%%%%%%%%%%%%%%%%%%%%%%%%%%%%%%%%%%%%%%%%%%%
% Subgrupo 3.2 - Coordenação de Eventos e Conferencista
%%%%%%%%%%%%%%%%%%%%%%%%%%%%%%%%%%%%%%%%%%%%%%%%%%%%%%%%%%%%%%%%%%%%%%%%%%%%%%%
\subsection{Coordenação de Eventos e Conferencista}
\vspace{0.3cm}

%------------------------------------------------------------------------------

\subsubsection{Comiss\~{a}o Organizadora de Eventos Internacional, Nacional, Regional ou Local}
\vspace{0.3cm}

\begin{enumerate}
\renewcommand{\labelenumi}{{\large\bfseries\arabic{enumi}.}}

    \item Organização do 11th Workshop on Software Modularity (WMod'14) \textbf{[Doc. \ref{org:2014-wmod}]}

    \item Organização do XXXV Congresso da Sociedade Brasileira de Computação (CSBC 2015) \textbf{[Doc. \ref{org:2015-csbc}]}

    \item Coordenador do \emph{7th International Workshop on Feature-Oriented Software Development (FOSD 2016)}. \textbf{[Doc. \ref{org:2016-fosd}]}

    \item \emph{Publicity Chair} do \emph{International Systems and Software Product Line Conference (SPLC 2016)}. \textbf{[Doc. \ref{org:2016-splc}]}

%Nada a declarar neste subgrupo.

\end{enumerate}

%%%%%%%%%%%%%%%%%%%%%%%%%%%%%%%%%%%%%%%%%%%%%%%%%%%%%%%%%%%%%%%%%%%%%%%%%%%%%%%
% Grupo 4: Atividades de Forma\c{c}\~{a}o e Capacita\c{c}\~{a}o Acad\^{e}mica
%%%%%%%%%%%%%%%%%%%%%%%%%%%%%%%%%%%%%%%%%%%%%%%%%%%%%%%%%%%%%%%%%%%%%%%%%%%%%%%
\newpage
\section{Atividades de Forma\c{c}\~{a}o e Capacita\c{c}\~{a}o Acad\^{e}mica}
\vspace{0.3cm}

%------------------------------------------------------------------------------

Nada a declarar neste subgrupo.
%
%\subsection{Atualiza\c{c}\~{a}o e Cursos de Capacita\c{c}\~{a}o ou Extens\~{a}o na \'{A}rea de Conhecimento ou Afins com no M\'{\i}nimo 40h}
%\vspace{0.3cm}
%
%\begin{enumerate}
%\renewcommand{\labelenumi}{{\large\bfseries\arabic{enumi}.}}
%
%\item   \textbf{Curso:}  Machine Learning (Coursera) \textbf{[Doc. \ref{courses:2015}]} \\
%        \textbf{Modalidade:} Ensino \`{a} dist\^{a}ncia (EAD) \\
%        \textbf{Carga Hor\'{a}ria:} 50 h  \\
%        \textbf{Instrutor:} Prof. Andrew Ng \\
%        \textbf{Institui\c{c}\~{a}o:} Stanford University
%
%\end{enumerate}


%%%%%%%%%%%%%%%%%%%%%%%%%%%%%%%%%%%%%%%%%%%%%%%%%%%%%%%%%%%%%%%%%%%%%%%%%%%%%%%
% Grupo 5: Atividades Administrativas
%%%%%%%%%%%%%%%%%%%%%%%%%%%%%%%%%%%%%%%%%%%%%%%%%%%%%%%%%%%%%%%%%%%%%%%%%%%%%%%
\newpage
\section{Atividades Administrativas}
\vspace{0.3cm}

%------------------------------------------------------------------------------

\subsection{Membro de Comiss\~{a}o Tempor\'{a}ria}
\vspace{0.3cm}

Nada a declarar neste subgrupo.
%\begin{enumerate}
%\renewcommand{\labelenumi}{{\large\bfseries\arabic{enumi}.}}
%
%\item   \textbf{Fun\c{c}\~{a}o:} Membro de comissão temporária \textbf{[Doc. \ref{committee:temp-2015}]}\\
%        \textbf{Comiss\~{a}o:} Presidente\\
%        \textbf{Per\'{\i}odo:} 01 a 15 de Outubro de 2015
%
%\end{enumerate}
%
Comissão seleção pós

%------------------------------------------------------------------------------

\subsection{Coordenador de Curso Pós-Graduação \textbf{strictu sensu} }
\vspace{0.3cm}

Nada a declarar neste subgrupo.
%
%\begin{enumerate}
%\renewcommand{\labelenumi}{{\large\bfseries\arabic{enumi}.}}
%
%\item   \textbf{Fun\c{c}\~{a}o:} Tutor da turma do Mestrado Profissional em Ciência da Computação com ênfase em Sistemas de Informação \textbf{[Doc. \ref{committee:postgrad-2015}]}\\
%        \textbf{Per\'{\i}odo:} 01 de Dezembro de 2014 a 01 de Dezembro de 2017.
%
%\end{enumerate}

%------------------------------------------------------------------------------

\subsection{Membro de Núcleo Docente Estruturante}
\vspace{0.3cm}

\begin{enumerate}
\renewcommand{\labelenumi}{{\large\bfseries\arabic{enumi}.}}

\item   \textbf{Fun\c{c}\~{a}o:} Membro do Núcleo Docente Estruturante do Curso de Graduação em Sistemas de Informação \textbf{[Doc. \ref{committee:nde-2015}]}\\
        \textbf{Per\'{\i}odo:} 21 de Julho de 2015 a 20 de Julho de 2016.

%\item   \textbf{Fun\c{c}\~{a}o:} Membro do Núcleo Docente Estruturante do Curso de Graduação em Sistemas de Informação \textbf{[Doc. \ref{committee:nde-2014}]}\\
%        \textbf{Per\'{\i}odo:} 21 de Julho de 2014 a 20 de Julho de 2015.

\end{enumerate}

%------------------------------------------------------------------------------

\subsection{Membro de Colegiados de Curso de Gradua\c{c}\~{a}o e P\'{o}s-Gradua\c{c}\~{a}o}
\vspace{0.3cm}

\begin{enumerate}
\renewcommand{\labelenumi}{{\large\bfseries\arabic{enumi}.}}

\item   \textbf{Fun\c{c}\~{a}o:} Membro do Colegiado da P\'{o}s-Gradua\c{c}\~{a}o \textbf{(em curso)}  \textbf{[Doc. \ref{committee:colegiado-postgrad-2015}]} \\
        \textbf{Per\'{\i}odo:} Desde Setembro de 2015.

\item   \textbf{Fun\c{c}\~{a}o:} Membro do Colegiado do Curso de Graduação em Ciência da Computação \textbf{[Doc. \ref{committee:colegiado-cc-2015}]} \\
        \textbf{Per\'{\i}odo:} 15 de Outubro de 2015 a 14 de Outubro de 2016.

\item   \textbf{Fun\c{c}\~{a}o:} Membro do Colegiado do Curso de Graduação em Sistemas de Informação \textbf{[Doc. \ref{committee:colegiado-si-2015}]} \\
        \textbf{Per\'{\i}odo:} 21 de Julho de 2015 a 20 de Julho de 2016.

\end{enumerate}

%COMPROVANTES!

%%%%%%%%%%%%%%%%%%%%%%%%%%%%%%%%%%%%%%%%%%%%%%%%%%%%%%%%%%%%%%%%%%%%%%%%%%%%%%%
% LISTA DE ANEXOS
%%%%%%%%%%%%%%%%%%%%%%%%%%%%%%%%%%%%%%%%%%%%%%%%%%%%%%%%%%%%%%%%%%%%%%%%%%%%%%%

\newpage
\section{Lista de Anexos}

A Tabela \ref{Tab:ListaAnexos} cont\'{e}m a numera\c{c}\~{a}o e uma pequena descri\c{c}\~{a}o dos documentos comprobat\'{o}rios em anexo. Quando for o caso, h\'{a} uma indica\c{c}\~{a}o entre par\^{e}nteses, ao final da descri\c{c}\~{a}o, do setor respons\'{a}vel pela emiss\~{a}o do documento.


\begin{table}[h]
\small
\caption{\texttt{Rela\c{c}\~{a}o numerada dos Anexos comprobat\'{o}rios}.}
\begin{tabular}{cl}
\toprule
\large{\textbf{\texttt{Doc}}} & \multicolumn{1}{c}{\large{\textbf{\texttt{Descri\c{c}\~{a}o}}}} \\
\otoprule
  1 & Declara\c{c}\~{a}o de orienta\c{c}\~{a}o de Mestrado de Maxwell Silva (PPGES/UPE). \\
  %\cmidrule{1-2}
  2 & Declara\c{c}\~{a}o de orienta\c{c}\~{a}o de Mestrado de Bruna Melo (PPGES/UPE). \\
  %\cmidrule{1-2}
  3 & Declara\c{c}\~{a}o de orienta\c{c}\~{a}o de TCC de Diocleciano Neto e Lucas Paes (Gradua\c{c}\~{a}o EC CIn). \\
  %\cmidrule{1-2}
  4 & Termo de orienta\c{c}\~{a}o de Monitoria de Lucas Harada - 2013.2 (PROACAD/UFPE). \\
  %\cmidrule{1-2}
  5 & Declara\c{c}\~{a}o de orienta\c{c}\~{a}o de Monitoria de Lucas Harada - 2013.1 (Sec Grad CIn). \\
  %\cmidrule{1-2}
  6 & Declara\c{c}\~{a}o de orienta\c{c}\~{a}o de IC de Djeefther Albuquerque (PROPESQ/UFPE). \\
  %\cmidrule{1-2}
  7 & Declara\c{c}\~{a}o de participa\c{c}\~{a}o em banca de Doutorado de Isaac Benchimol (PPGEE/UFPE). \\
  %\cmidrule{1-2}
  8 & Declara\c{c}\~{a}o de participa\c{c}\~{a}o em banca de Doutorado de Walter Guimar\~{a}es (PPGEE/UFPE). \\
  \cmidrule{1-2}
  9 & Declara\c{c}\~{a}o de participa\c{c}\~{a}o em banca de Mestrado de Joyce Teixeira (PGCC/UFPE). \\
    & C\'{o}pia da Portaria PGCC 166/2012 - Composi\c{c}\~{a}o de banca de Mestrado (PGCC/UFPE). \\
  \cmidrule{1-2}
  10 & Declara\c{c}\~{a}o de participa\c{c}\~{a}o em banca de Mestrado de Antonio Assun\c{c}\~{a}o (PPGES/UPE). \\
     & C\'{o}pia da Portaria PPGES 11/2013 - Composi\c{c}\~{a}o de banca de Mestrado (PPGES/UPE). \\
  \cmidrule{1-2}
  11 & Declara\c{c}\~{a}o de participa\c{c}\~{a}o em banca de Mestrado de Maxwell Silva (PPGES/UPE). \\
     & C\'{o}pia da Portaria PPGES 15/2013 - Composi\c{c}\~{a}o de banca de Mestrado (PPGES/UPE). \\
  \cmidrule{1-2}
  12 & Declara\c{c}\~{a}o de participa\c{c}\~{a}o em banca de Mestrado de Bruna Melo (PPGES/UPE). \\
     & C\'{o}pia da Portaria PPGES 16/2013 - Composi\c{c}\~{a}o de banca de Mestrado (PPGES/UPE). \\
  \cmidrule{1-2}
  13 & Declara\c{c}\~{a}o de participa\c{c}\~{a}o em banca de TCC (Gradua\c{c}\~{a}o-EC CIn). \\
  %\cmidrule{1-2}
  14 & Declara\c{c}\~{a}o de particip. em banca de Sele\c{c}\~{a}o de Prof. Tempor\'{a}rio do CIn/UFPE (Dire\c{c}\~{a}o CIn). \\
  %\cmidrule{1-2}
  15 & Relat\'{o}rio de disciplinas lecionadas na gradua\c{c}\~{a}o no per\'{\i}odo de avalia\c{c}\~{a}o (SIGA-UFPE).\\
  %\cmidrule{1-2}
  16 & C\'{o}pia do certificado de participa\c{c}\~{a}o no XXX SBrT. \\
  %\cmidrule{1-2}
  17 & C\'{o}pia do certificado de participa\c{c}\~{a}o no XXXI SBrT. \\
  %\cmidrule{1-2}
  18 & C\'{o}pias do resumo publicado e do e-mail de aceita\c{c}\~{a}o de trabalho no XXXI SBrT. \\
  %\cmidrule{1-2}
  19 & C\'{o}pias do artigo publicado e do e-mail de aceita\c{c}\~{a}o de trabalho no XXX SBrT. \\
  %\cmidrule{1-2}
  20 a 23 & C\'{o}pias dos artigos publicados e dos e-mails de aceita\c{c}\~{a}o de trabalho no CBA 2012 e XXXI SBrT. \\
  %\cmidrule{1-2}
  24 a 31 & C\'{o}pias dos e-mails de convite para revis\~{a}o e finaliza\c{c}\~{a}o do processo (peri\'{o}dicos e eventos). \\
  \cmidrule{1-2}
  32 & C\'{o}pia do Resultado do Edital 10/2010 (pags 1,2 e 6) e Termo de Outorga do projeto (FACEPE). \\
     & C\'{o}pia de correspond\^{e}ncia comprobat\'{o}ria de presta\c{c}\~{a}o de contas do projeto (FACEPE). \\
  \cmidrule{1-2}
  33 & C\'{o}pias de e-mail de convite para avaliador e de cabe\c{c}alho do parecer emitido (FACEPE). \\
  %\cmidrule{1-2}
  34 & C\'{o}pia do certificado de premia\c{c}\~{a}o de melhor artigo de IC do XXXI SBrT. \\
  %\cmidrule{1-2}
  35 & C\'{o}pias do artigo publicado no peri\'{o}dico TEMA e do e-mail de aceita\c{c}\~{a}o. \\
  %\cmidrule{1-2}
  36 & C\'{o}pias do artigo publicado no peri\'{o}dico EURASIP J. Adv. Sig. Proc. e do e-mail de aceita\c{c}\~{a}o. \\
  %\cmidrule{1-2}
  37 & C\'{o}pias dos e-mails comprobat\'{o}rios de convite e resposta ao convite (MIC-CSC2012). \\
  %\cmidrule{1-2}
  38 & C\'{o}pias dos e-mails comprobat\'{o}rios de convite e resposta ao convite (XXXI SBrT). \\
  %\cmidrule{1-2}
  39 & C\'{o}pia da declara\c{c}\~{a}o enviada \`{a} Comiss\~{a}o Organizadora da SBPC 2013 (Dire\c{c}\~{a}o CIn). \\
  %\cmidrule{1-2}
  40 & Online Course Statement of Accomplishment: Machine Learning - Stanford University - Coursera. \\
  %\cmidrule{1-2}
  41 - 42 & Declara\c{c}\~{o}es de participa\c{c}\~{a}o na Comiss\~{a}o de Sele\c{c}\~{a}o do Mestrado CIn - 2012 e 2013. \\
  %\cmidrule{1-2}
  43 & Declara\c{c}\~{a}o comprobat\'{o}ria de cargo ocupado - Subcoordena\c{c}\~{a}o de Editais. \\
  %\cmidrule{1-2}
  44 & Declara\c{c}\~{a}o de participa\c{c}\~{a}o no Colegiado da P\'{o}s-Gradua\c{c}\~{a}o do CIn. \\
  %\cmidrule{1-2}
  45 & C\'{o}pia da Portaria de Designa\c{c}\~{a}o 005/2013 - Colegiado Gradua\c{c}\~{a}o EC CIn (Dire\c{c}\~{a}o CIn). \\
\bottomrule
\end{tabular}
\label{Tab:ListaAnexos}
\end{table}

% Appendix
\clearpage
%\addappheadtotoc
\appendix
%\appendixpage
\newpage
\section{Documentos comprobatórios}
Esta seção contém os documentos comprobatórios referentes às atividades listadas neste memorial.
\addcontentsline{toc}{section}{Documentos comprobatórios}
\renewcommand{\thesubsection}{\arabic{subsection}}
% \renewcommand{\subsection}{
% \titleformat{\subsection}
%   {\Huge\bfseries\center\vspace{.4\textwidth}\thispagestyle{fancy}} % format
%   {}                % label
%   {0pt}             % sep
%   {\huge}           % before-code
% }

\include{appendices}


\end{document}

%%% EOF
